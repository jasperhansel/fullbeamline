\chapter{Reading and Writing Lattices}

%----------------------------------------------------------------------------
\section{Reading in Lattices}
\label{s:lat.readin}
\index{lattice files!reading}

\index{XSIF}\Hyperref{r:xsif.parser}{xsif_parser}
\Hyperref{r:bmad.parser}{bmad_parser}\Hyperref{r:bmad.parser2}{bmad_parser2}

\bmad has routines for reading XSIF (\sref{s:lattice.file.formats}) and
\bmad formatted lattice files. The subroutine to read in an XSIF lattice
file is \Hyperref{r:xsif.parser}{xsif_parser}. There are two subroutines in \bmad to read
in a \bmad standard lattice file: \Hyperref{r:bmad.parser}{bmad_parser} and
\Hyperref{r:bmad.parser2}{bmad_parser2}. \vn{bmad_parser} is used to initialize a
\vn{lat_struct} (\sref{c:lat.struct}) 
structure from scratch using the information from a
lattice file. Unless told otherwise, after reading in the lattice,
\vn{bmad_parser} will compute the 6x6 transfer matrices for each element
and this information will be stored in the \vn{digested file}
(\sref{s:digested}) that is created.  Notice that \vn{bmad_parser}
does {\em not} compute any Twiss parameters.

\Hyperref{r:bmad.parser2}{bmad_parser2} is typically used after \vn{bmad_parser} if there is
additional information that needs to be added to the lattice. For
example, consider the case where the aperture limits for the elements 
is stored in a file that is separate from the main lattice definition
file and it is undesireable to put a \vn{call} statement in one file
to reference the other.
To read in the lattice information along with the aperture limits, 
there are two possibilities: One possibility 
is to create a third file that calls the first two:
\begin{verbatim}
 ! This is a file to be called by bmad_parser
 call, file = 'lattice_file'
 call, file = 'aperture_file'
\end{verbatim}
and then just use \vn{bmad_parser} to parse this third file. The
alternative is to use \vn{bmad_parser2} so that the program code looks
like:
\begin{verbatim}
  ! program code to read in everything
  type (lat_struct) lat
  call bmad_parser ('lattice_file', lat)       ! read in a lattice.
  call bmad_parser2 ('aperture_file', lat)     ! read in the aperture limits.
\end{verbatim}

An alternative to using \vn{bmad_parser} and \vn{xsif_parser} is to
use the combined \bmad and XSIF parser
\Hyperref{r:bmad.and.xsif.parser}{bmad_and_xsif_parser}. This parser
will assume that the input file is using \bmad syntax unless the file
name is prefixed by the string \vn{``xsif::''}.

%----------------------------------------------------------------------------
\section{Digested Files}
\index{digested files}

\index{bmad_parser}
\index{bmad version number}
Since parsing can be slow, once the \vn{bmad_parser} routine
has transfered the
information from a lattice file into the \vn{lat_struct} it will make
what is called a digested file. A digested file is an image of the
\vn{lat_struct} in binary form. When \vn{bmad_parser} is called, 
it first looks in the same directory as the lattice
file for a digested file whose name is of the form:
\begin{verbatim}
  'digested_' // LAT_FILE 
\end{verbatim}
where \vn{LAT_FILE} is the lattice file name. If \vn{bmad_parser} finds the digested
file, it checks that the file is not out--of--date (that is, whether the
lattice file(s) have been modified after the digested file is made).
\vn{bmad_parser} can do this since the digested file contains the names
and the dates of all the lattice files that were involved. Also stored
in the digested file is the ``\bmad \vn{version number}''. The \bmad
version number is a global parameter that is increased (not too
frequently) each time a code change involves modifying the structure of
the \vn{lat_struct} or \vn{ele_struct}. If the \bmad version number in
the digested file does not agree with the number current when \vn{bmad_parser}
was compiled, or if the digested
file is out--of--date, a warning will be printed, and \vn{bmad_parser}
will reparse the lattice and create a new digested file.

\index{taylor map!with digested files}
Since computing Taylor Maps can be very time intensive,
\vn{bmad_parser} tries to reuse Taylor Maps it finds in the digested
file even if the digested file is out--of--date. To
make sure that everything is OK, \vn{bmad_parser} will check that the attribute
values of an element needing a Taylor map are the same as the
attribute values of a corresponding element in the digested file
before it reuses the map. Element names are not a factor in this
decision.

This leads to the following trick: If you want to read in a lattice
where there is no corresponding digested file, and if there is another
digested file that has elements with the correct Taylor Maps, then, to
save on the map computation time, simply make a copy of the digested
file with the digested file name corresponding to the first lattice.

\Hyperref{r:read.digested.bmad.file}{read_digested_bmad_file}
\Hyperref{r:write.digested.bmad.file}{write_digested_bmad_file}
The digested file is in binary format and is not human readable but it
can provide a convenient mechanism for transporting lattices between
programs. For example, say you have read in a lattice, changed
some parameters in the \vn{lat_struct}, and now you want to do some
analysis on this modified \vn{lat_struct} using a different program. 
One possibility is to have the first program create a digested file
\begin{example}
  call write_digested_bmad_file ('digested_file_of_mine', lat)
\end{example}
and then read the digested file in with the second program
\begin{example}
  call read_digested_bmad_file ('digested_file_of_mine', lat)
\end{example}
An alternative to writing a digested file is to write a lattice file
using \vn{write_bmad_lattice_file}

%----------------------------------------------------------------------------
\section{Writing Lattice files}
\label{s:lat.write}
\index{lattice files!MAD files}
\index{MAD}

\Hyperref{r:write.bmad.lattice.file}{write_bmad_lattice_file}
To create a \bmad lattice file from a \vn{lat_struct} instance, use the routine
\Hyperref{r:write.bmad.lattice.file}{write_bmad_lattice_file}.
\index{MAD!MAD-8}
\mad--8, \mad--X, or \vn{XSIF} compatible lattice 
files can be created from a \vn{lat_struct} variable 
using the routine 
\Hyperref{r:write.lattice.in.foreign.format}{write_lattice_in_foreign_format}:
\begin{example}
  type (lat_struct) lat             ! lattice
  ...
  call bmad_parser (bmad_lat_file, lat)               ! Read in a lattice
  call write_lattice_in_foreign_format ('lat.mad', 'MAD-8', lat)  ! create MAD file
\end{example}
Information can be lost when creating a \mad or \vn{XSIF} file.
For example, neither \mad nor \vn{XSIF} has the concept of 
things such as \vn{overlay}s and \vn{group}s.

