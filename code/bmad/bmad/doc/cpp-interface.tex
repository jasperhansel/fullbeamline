\chapter{C++ Interface}
\label{c:cpp.interface}
\index{C++ interface}

To ease the task of using \cpp routines with \bmad, there is a
library called \vn{cpp_bmad_interface} which implements a set of \cpp
classes in one--to--one correspondence with the major \bmad
structures. In addition to the \cpp classes, the \bmad library
defines a set of conversion routines to transfer data values between
the \bmad Fortran structures and the corresponding \cpp classes.

The list of all classes is given in the file
\begin{example}
  cpp_bmad_interface/include/cpp_bmad_classes.h
\end{example}
The general rule is that the equivalent class to a \bmad structure
named \vn{xxx_struct} will be named \vn{CPP_xxx}. Additionally, for
each \bmad structure, there is a opaque class named \vn{Bmad_xxx_class}
for use in the translation code discussed below. The names of these
opaque classes have the form \vn{Bmad_xxx_class} and are used to define
pointer instances in routine argument lists.

%----------------------------------------------------------------------------
\section{C++ Classes and Enums}
\index{C++ interface!classes}

Generally, The \cpp classes have been set up to simply mirror the
corresponding \bmad structures. For example, the \vn{CPP_lat} class
has a string component named \vn{.version} that mirrors the
\vn{%version} component of the \vn{lat_struct} structure. There are
some exceptions. For example, structure components that are part of
\vn{PTC} (\sref{s:ptc.intro}) are not present in the classes.

While generally the same component name is used for both the \bmad
structures and the \cpp classes, in the case where there is a \cpp
reserved word conflict, the \cpp component name will be different.

A header file \vn{bmad_enums.h} defines corresponding \bmad
parameters for all \cpp routine. The \bmad parameters are in a
namespace called \vn{Bmad}. The convention is that the name of a
corresponding \cpp parameter is obtained by dropping the ending
\vn{\$} (if there is one) and converting to uppercase. For example,
\vn{electron\$} on the Fortran side converts to \vn{Bmad::ELECTRON} in
\cpp. 

All of the \cpp class components that are arrays or matrices are zero
based so that, for example, the index of the \vn{.vec[i]} array in a
\vn{CPP_coord} runs from 0 through 5 and not 1 through 6 as on the
Fortran side. Notice that for a \vn{lat_struct} the \vn{%ele(0:)}
component has a starting index of zero so there is no off--by--one
problem here.  The exception to this rule is the \vn{%value(:)} array
of the \vn{ele_struct} which has a span from 1 to
\vn{num_ele_attrib\$}. In this case, To keep the conversion of the of
constructs like \vn{ele%value(k1\$)} consistant, the corresponding
\vn{ele.value[]} array has goes from 0 to \vn{Bmad::NUM_ELE_ATTRIB}
with the 0th element being unused.

%----------------------------------------------------------------------------
\section{Conversion Between Fortran and C++}
\index{C++ interface!Fortran calling C++}

\begin{figure}[tb]
\begin{listing}{1}
  subroutine f_test
    use bmad_cpp_convert_mod
    implicit none

    interface
      subroutine cpp_routine (f_lat, c_coord) bind(c)
        import f_lat, c_ptr
        type (lat_struct) :: f_lat
        type (c_ptr), value :: c_coord
      end subroutine
    end interface

    type (lat_struct), target :: lattice   // lattice on Fortran side 
    type (coord_struct), target :: orbit
    type (c_ptr), value :: c_lat
    ! ... 
    call lat_to_c (c_loc(lattice), c_lat)    ! Fortran side convert
    call cpp_routine (c_lat, c_loc(orbit))   ! Call C++ routine
    call lat_to_f (c_lat, c_loc(lattice))    ! And convert back
  end subroutine
\end{listing}
\caption{Example Fortran routine calling a \cpp routine.}
\label{f:fortran}
\end{figure}


\begin{figure}
\begin{listing}{1}
  #include "cpp_bmad_classes.h"

  using namespace Bmad;

  extern "C" cpp_routine (CPP_lat& c_lat, Bmad_coord_class* f_coord,  f_lat) {
    CPP_coord c_coord;
    coord_to_c (f_coord, c_coord);        // C++ side convert
    // ... do calculations ...
    cout << c_lat.name << "  " << c_lat.ele[1].value[K1] << endl;
    coord_to_f (c_coord, f_coord);        // And convert back
  }
\end{listing}
\caption{Example \cpp routine callable from a Fortran routine.}
\label{f:cpp}
\end{figure}

A simple example of a Fortran routine calling a \cpp routine is shown
in \figs{f:fortran} and \ref{f:cpp}. Conversion between structure and
classes can happen on either the Fortran side or the \cpp side. In
this example, the \vn{lat_struct} / \vn{CPP_lat} conversion is on the
Fortran side and the \vn{coord_struct} / \vn{CPP_coord} is on the \cpp
side. 

On the Fortran side, the interface block defines the argument list of
the \cpp routine being called.

On the \cpp side, \vn{f_coord} is an instance of the
\vn{Bmad_coord_class} opaque class.

A \cpp routine calling a Fortran routine has a similar structure to
the above example. The interface block in \fig{f:fortran} can be used
as a prototype. For additional examples of conversion between Fortran
and \cpp, look at the test code in the directory
\begin{example}
  cpp_bmad_interface/interface_test
\end{example}
